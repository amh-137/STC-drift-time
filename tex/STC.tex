\documentclass[11pt]{article}

\usepackage{tex/STC}

%Document
\begin{document}

\title{STC Drift-Time Relation\\\normalsize Computing Project}
\author{Alex Hergenhan\\CID: 0604432}
\maketitle

This report outlines a program written to study the drift-time relationship for gasses in the Small Prototype Test Chamber (STC) for the ALEPH Inner Tracking Chamber (ITC).

\section{Introduction}
\label{sec:intro}

\section{Method 1 - Minimisation \& Tangents}
\label{sec:method}
I utilize a minimisation algorithm with some basic geometry to find the velocities of ionisation electrons. This works as follows:
\begin{itemize}
    \item The two largest circles by TDC time are picked $c_1$ and $c_2$.
    \item The four tangents for these two circles are caluclated using \texttt{???}.
    \item The sum of the squares of the distances to the other 6 circles are found (the distance to the two largest circles is zero as these are tangents).
    \item The radii of the circles is varied by multiplying by a constant factor $v$ and the sum of the squares of the distances is recalculated.
\end{itemize}
The factor $v$ is varied until the sum of the squares of the distances is minimised. This is the velocity of the ionisation electrons.

\subsection{Tangents \& Circle - Line Distances}
\label{sec:tangents}
We define circles and lines by their equations
\begin{equation}
    (x - x_0)^2 + (y - y_0)^2 = r^2
    \label{eq:circle}
\end{equation}
\begin{equation}
    ax + by + c = 0,
    \label{eq:line}
\end{equation}
The parameters of these equations ($a$, $b$, $c$, $x_0$, $y_0$, $r$) are stored in the structs \texttt{line} and \texttt{circle}. The tangents are therefore lines with
\begin{equation}
    a = 
    b = 
    c = .
    \label{eq:tangents}
\end{equation}
Finally, the distance between the $i$th circle and a line is given by
\begin{equation}
    d_i = 
    \label{eq:dist}
\end{equation}

\subsection{Minimisation}
\label{sec:minimisation}
The loss function used is defined as:
\begin{equation}
    L = \sum\limits_{\substack{i=1 \\ i\neq c_1, c_2}}^{8} d_i^2
    \label{eq:loss}
\end{equation}
This is minimised with gradient descent. The gradient of this function could be caluclated analytically, however we used the definition of the derivative.
\begin{equation}
    \frac{df}{dv} = \frac{f(v + \delta v) - f(v)}{\delta v}
    \label{eq:grad}
\end{equation}
In the function \texttt{???}, iterate
\begin{equation}
    v_{j+1} = v_j - \eta \frac{df}{dv}\Bigr|_{v_j}
    \label{eq:iter}
\end{equation}


\section{Method 2 - Linear Regression \& Invarient Points}
\label{sec:linreg}



\end{document}